\beginsong{Marie}[by={\normalsize Tomáš Klus}]
\beginverse
Je \[F]den, tak \[A]pojď Marie ven,
budeme \[B]žít, házet š\[C]utry do oken.
Je \[F]dva - necháme \[A]doma trucovat,
když nechtě\[B]jí - nemusí, nebud\[C]em se vnucovat.
Jémi\[F]ne, všechno \[A]zlý jednou pomine,
tak \[B]Marie, c\[C]o ti je?
\endverse

\beginverse
\[F]\[A]\[B]\[C]\vspace{-0.5cm}
\endverse

\beginverse
Všemoc\[F]né - jsou \[A]loutkařovy prsty,
ať jsou \[B]tenký nebo tlustý, občas p\[C]řetrhají nit.
A to pak \[F]jít, a nemít \[A]nad sebou svý jistý,
pořád \[B]s tváří optimisty, listy v\[C] žití obracet.
Je to \[F]jed, mazat si \[A]kolem huby med
a nesly\[B]šet, jak se ti b\[C]ortí svět.
Mari\[F]e, kdo přeží\[A]vá nežije,
tak ádi\[B]jé.\[C]
\endverse
\beginverse
\[F]Marie, už zase \[A]máš (k) tulení sklony,
jako \[B]loni, slyším k\[C]ostelní zvony znít.
\[F]A to mě zabije, \[A]a to mě zabije,
\[B]a to mě zabije, j\[C]istojistě.
\endverse

\beginchorus
Já \[F]mám Marie \[A]rád, když má\[B] moje bytí s\[C]pád,
býti \[F]věčně na ces\[A]tách a k ránu \[B]spícím plícím ž\[C]ivot vdechovat.
\[F]Nechtěj mě milovat, \[A]nechtěj mě milovat, \[B]nechtěj mě milova\[C]t.
Já \[F]mám Marie \[A]rád, když má\[B] moje bytí s\[C]pád,
býti \[F]věčně na ces\[A]tách a k ránu \[B]spícím plícím ž\[C]ivot vdechovat.
\endchorus
\beginverse
\[F]\[A]\[B]\[C] \vspace{-0.5cm}
\endverse

\beginverse
Copak \[F]nemůže být mezi \[A]ženou a mužem
\[B]přatelství, kde není n\[C]ikdo nic dlužen.
\[F]Prostě jen prosté \[A]spříznění duší,
\[B]aniž by kdokoliv c\[C]okoliv tušil.
\endverse

\beginverse
\[F]Na n\[A]a na\[B] na..\[C].
\endverse

\refchorus

\beginverse
\[F]\[A]\[B]\[C] \vspace{-0.5cm}
\endverse
\endsong