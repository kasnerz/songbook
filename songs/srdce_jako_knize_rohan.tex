\beginsong{Srdce jako kníže Rohan}[by={\normalsize Richard Müller}]
\beginverse
   \[F]Měsíc je jak Zlatá bula \[C]Sicilská
   \[A&]Stvrzuje že kdo chce ten se \[G]dopíská
   \[F]pod lampou jen krátce v přítmí \[C]dlouze zas
   \[A&]Otevře ti Kobera a \[G]můžeš mezi \[F]nás.\[A&]\[G]\[F]\[C]\[A&]\[G]
\endverse

\beginverse
   \[F]Moje teta, tvoje teta, \[C]parole
   \[A&]dvaatřicet karet křepčí \[G]na stole
   \[F]měsíc svítí sám a chleba \[C]nežere
   \[A&]Ty to ale koukej trefit \[G]frajere. Protože
\endverse

\beginchorus
   \[F]Dnes je valcha u starýho \[C]Růžičky
   \[A&]dej si prachy do pořádny \[G]roličky
   \[F]Co je na tom že to není \[C]extra nóbl byt
   \[A&]Srdce jako kníže Rohan \[G]musíš mít.\[C]\[A&]\[G]
\endchorus

\beginverse
   ^Ať si přes den docent nebo ^tunelář
   ^herold svatý pravdy nebo ^jinej lhář
   ^tady na to každej kašle ^zvysoka
   ^pravda je jen jedna - ^slova proroka říkaj že:
\endverse

\beginchorus
   ^Když je valcha u starýho ^Růžičky
   ^budou v celku nanic všechny ^řečičky
   ^Buď to trefa nebo kufr - ^smůla nebo šnyt
   ^jen to srdce jako Rohan ^musíš mít.
\endchorus

\beginverse
   ^Kdo se bojí má jen hnědý ^kaliko
   ^možná občas nebudeš mít ^na mlíko
   ^jistě ale poznáš ^co jsi vlastně zač
   ^svět nepatřil nikomu kdo ^nebyl hráč. A proto
\endverse

\beginchorus
   ^Ať je valcha u starýho ^Růžičky
   ^nebo pouť až k tváři Boží ^rodičky
   ^Ať je válka, červen, mlha, ^bouřka nebo klid
   ^Srdce jako kníže Rohan ^musíš mít.
\endchorus

\beginchorus
   ^Dnes je valcha u starýho ^Růžičky
   ^když si malej tak si stoupni ^na špičky
   ^malej nebo nachlapenej ^cikán, prdák, žid
   ^Srdce jako kníže Rohan ^musíš mít.
\endchorus

\beginchorus
   ^Dnes je valcha u starýho ^Růžičky
   ^dej si prachy do pořádny ^roličky
   ^Co je na tom že to není ^extra nóbl byt
   ^Srdce jako kníže Rohan ^musíš mít.
\endchorus
\endsong
