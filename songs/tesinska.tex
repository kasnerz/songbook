\beginsong{Těšínská}[by={\normalsize Jaromír Nohavica}]
\beginverse
\[A&]Kdybych se narodil \[D&]před sto lety\[F] v \[E]tomhle \[A&]měst\[D&]ě,\[F]\[E]\[A&]
u Larischů na zahradě \[D&]trhal bych květy\[F] s\[E]vé ne\[A&]věst\[D&]ě,\[F]\[E]\[A&]
\[C]moje nevěsta by \[D&]byla dcera ševcova
z \[F]domu Kamiňskich \[C]odněkud ze Lvova,
kochal bym ja i \[D&]pieščil,\[F] ch\[E]yba lat \[A&]dwieščie.\[D&]\[F]\[E]\[A&]
\endverse

\beginverse
\[A&]Bydleli bychom na \[D&]Sachsenbergu \[F]v domě u \[E]žida \[A&]Kohn\[D&]a,\[F]\[E]\[A&]
nejhezčí ze všech \[D&]těšínských šperků\[F] by\[E]la by \[A&]ona,\[D&]\[F]\[E]\[A&]
\[C]mluvila by polsky \[D&]a trochu česky,
\[F]pár slov německy, a \[C]smála by se hezky,
jednou za sto let \[D&]zázrak se koná,\[F] zá\[E]zrak se \[A&]koná\[D&].\[F]\[E]\[A&]
\endverse

\beginverse
\[A&]Kdybych se narodil \[D&]před sto lety, \[F]byl bych \[E]vazačem \[A&]knih\[D&]\[F]\[E]\[A&]
u Prohazků dělal bych od \[D&]pěti do pěti a \[F]sedm zlatek \[E]za to \[A&]bral bych\[D&],\[F]\[E]\[A&]
\[C]měl bych krásnou \[D&]ženu a tři děti,
\[F]zdraví bych měl a bylo \[C]by mi kolem třiceti,
celý dlouhý život \[D&]před sebou, celé \[F]krásné \[E]dvacáté \[A&]století\[D&].\[F]\[E]\[A&]
\endverse

\beginverse
\[A&]Kdybych se narodil \[D&]před sto lety\[F] v ji\[E]načí \[A&]době\[D&],\[F]\[E]\[A&]
u Larischů na zahradě \[D&]trhal bych květy,\[F] má \[E]lásko, \[A&]tobě\[D&],\[F]\[E]\[A&]
\[C]tramvaj by jezdila \[D&]přes řeku nahoru,
\[F]slunce by zvedalo \[C]hraniční závoru
a z oken \[D&]voněl by\[F] svá\[E]teční \[A&]oběd\[D&].\[F]\[E]\[A&]
\endverse

\beginverse
\[A&]Večer by zněla \[D&]od Mojzese \[F]melodie \[E]dávno\[A&]věká\[D&],\[F]\[E]\[A&]
bylo by léto tisíc \[D&]devět set deset, \[F]za domem by \[E]tekla \[A&]řeka\[D&],\[F]\[E]\[A&]
\[C]vidím to jako dnes: \[D&]šťastného sebe,
\[F]ženu a děti a \[C]těšínské nebe,
ještě že člověk \[D&]nikdy neví\[F], \[E]co ho \[A&]čeká,
\endverse

\beginverse
\[D&]na n\[F]a \[E]na\[A&] ...\[D&]\[F]\[E]\[A&]
\endverse
\endsong