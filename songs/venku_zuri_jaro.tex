\beginsong{Venku zuří jaro}[by={\normalsize Jarret}]
\beginverse
\[E]Sleduju jak jeden pán u okýnka usíná,
\[F#&]klimbá a slintá něco se mu z\[E]dá.
S trhnutím pak procitá a jak \[F#&]tráva deštěm pobitá,
vstává \[G#&]před ústa si \[F#&]ruku \[A]dává něka\[H]m utí\[E]ká.
\endverse

\beginverse
\[E]Na jeho místo k oknu usedá žena, kterou dobře znám
\[F#&]z obálky Květů, jež držel tamten \[E]pán.
Do skla na sebe se podívá a \[F#&]hlavou smutně pokývá.
\[G#&]Ví že už jen \[F#&]musí, že \[A]žádný můžeš \[H]nepla\[E]tí.
\endverse

\beginverse
A ona \[H]ví\[C#&],    že   \[A]nehřeje že nechla\[E]dí
a \[A]na dotek je jak se \[H]zdá tak ako\[E]rát.
\endverse

\beginchorus
Venku zuří jaro, \[A]já mám jednu starost,
kdy ví\[E]celetku zasadím.
Chce to ňákou změnu,
\[A]to co nedoženu,
\[E]z dohledu snad neztratím.
\endchorus

\beginverse
\[E]Kam zmizel onen spící muž, co ho bodá jako nůž,
\[F#&]nůž ostrý jako kletba, jako uráž\[E]ka.
Krátký sen uprostřed dne ho \[F#&]nechá, ať si vzpomene
\[G#&]na to, na co \[F#&]myslet \[A]nechce, ale \[H]spíše \[E]ne.
\endverse

\beginverse
\[E]A ta žena možná ví, že ví, ale stejně z toho nesleví,
\[F#&]už je přece pozdě, už nic ne\[E]změní.
Zpívám si spolu s ní o \[F#&]věcech prvních posledních,
\[G#&]místo tečky, \[F#&]bože, \[A]chtěl bych radši \[H]vykřič\[E]ník.
\endverse

\beginverse
I když \[H]ví\[C#&]m,   že \[A]nehřeju, že nechla\[E]dím,
a \[A]na dotek jsem, jak se \[H]zdá, tak ako\[E]rát.
\endverse

\refchorus
\endsong